%*******************************************************************************
%  Zadání:
%  dokumentace.pdf      4-7 stran
%  1. strana            OK
%  Diagram automatu    OK
%  LL-gramatika        OK
%  prece tabulka       "OK"
%  POPIS ZPŮSOBU ŘEŠENÍ Z POHLEDU IFJ
%      návrh
%      implementace
%      vývojový cyklus
%      práce v týmu
%      sleciální techniky
%      algoritmy
%  Popis řadícího svého algoritmu
%      řazení
%      hledání v textu
%      tabulky symbolů
%  Opět práce v týmu (kdo co jak)
%  Literatutra, reference, citace
%  !! Nevlastní materiál
%  !! Obecné popisy algoritmů

%  Pavel:
%  kontrola čitelnosti kódu
%  kontrola komentovanosti kódů, zdroje! (např. odkud prvočísla do hash tabulky)
%  kontrola prece tabulky X změna za symboly 	textless > = !
%  přidat dolar do prece tabulky
%*******************************************************************************


%
% Course IFJ @ FIT VUT Brno, 2015
% IFJ15 Interpreter Project
%
% Authors:
% Lukas Osadsky  - xosads00
% Pavol Plaskon  - xplask00
% Pavel Pospisil - xpospi88
% Matej Postolka - xposto02

\documentclass[a4paper, 12pt]{article}

% Použité nbalíčky
\usepackage[left=2cm,text={17cm, 24cm},top=3cm]{geometry}
\usepackage[czech]{babel}
\usepackage[utf8]{inputenc}
\usepackage[IL2]{fontenc}
%\usepackage{times}
\usepackage{graphics}
\usepackage{url}
% Požadavky automatu
\usepackage{pgf}
\usepackage{tikz}
\usetikzlibrary{arrows,automata}
% Požadavky tabulky
\usepackage{diagbox}
% Rotace stránky
\usepackage{pdflscape}

\pagenumbering{arabic}
\providecommand{\uv}[1]{\quotedblbase #1\textquotedblleft}
\interlinepenalty 10000 %nezalamovat odstavce

\begin{document}
%%%%%%%%%%%%%%%%%%%%%%%%%%%%%%%%%%%%%%%%%%%%%%%%
%               Title page
%%%%%%%%%%%%%%%%%%%%%%%%%%%%%%%%%%%%%%%%%%%%%%%%
\begin{titlepage}

\begin{center}
\fontsize{25}{20}\selectfont{\textsc{Vysoké učení technické v~Brně}}\\
\vspace{\stretch{0.0075}}
\fontsize{21}{0}\textsc{\selectfont{Fakulta informačních technologií}}\\
\vspace{\stretch{0.15}}
%Logo
\begin{figure}[ht]
    \begin{center}
        \scalebox{1}{\includegraphics{logo_bw.eps}}
    \end{center}
\end{figure}

\vspace{\stretch{0.2}}
\LARGE{Dokumentace IFJ15}\\
\Large{Tým \textit{052}, varianta \textit{a/2/II}}
\vspace{\stretch{0.618}}
\end{center}

\begin{large}
\begin{tabbing}
    Další členové: \quad \= Postolka Matej \quad \= xposto02 \quad \= 25 \%\kill
    Vedoucí týmu:  \> Postolka Matěj \> xposto02 \> 25 \% \\
    Další členové: \> Osadský Lukáš  \> xosads00 \> 25 \% \\
             \> Plaskoň Pavol  \> xplask00 \> 25 \% \\
             \> Pospíšil Pavel \> xpospi88 \> 25 \% \\
\end{tabbing}
\end{large}

\end{titlepage}

%%%%%%%%%%%%%%%%%%%%%%%%%%%%%%%%%%%%%%%%%%%%%%%%
%                   Obsah
%%%%%%%%%%%%%%%%%%%%%%%%%%%%%%%%%%%%%%%%%%%%%%%%
\tableofcontents
\newpage
%%%%%%%%%%%%%%%%%%%%%%%%%%%%%%%%%%%%%%%%%%%%%%%%
%                   Úvod
%%%%%%%%%%%%%%%%%%%%%%%%%%%%%%%%%%%%%%%%%%%%%%%%
\section{Úvod} \label{uvod}

Tato dokumentace popsiuje implementaci interpretu jazyka IFJ15,který je
zjednodušenou podmnožinou jazyka C++11. Interpret se skládá ze čtyřech částí
popsaných v následujících kapitolách.

\begin{itemize}
	\item Lexikální analyzátor
	\item Syntaktický analyzátor
	\item Sémantický analyzátor
	\item Interpret
\end{itemize}
%%%%%%%%%%%%%%%%%%%%%%%%%%%%%%%%%%%%%%%%%%%%%%%%
%                   Práce v týmu
%%%%%%%%%%%%%%%%%%%%%%%%%%%%%%%%%%%%%%%%%%%%%%%%
\section{Práce v týmu} \label{team}

\subsection{Rozdělení práce na jednotlivých částech}
\begin{itemize}
	\item Lexer Lukáš -- lexikální analyzátor
	\item Matěj Postolka -- Sémantický a syntaktický analyzátor
	\item Pavel Pospíšil -- Zpracování výrazů a volání funkcí
	\item Pavel Plaskoň -- Interpret, vstavěné funkce
\end{itemize}

\subsection{Průběh vývoje}
Projekt je řešený čtyřčlenným týmem, bylo tedy potřebné zvolit vhodný systém
správy zdrojových souborů. Přes téměř nulové zkušenosti většiny členů týmu jsme
k těmto účelům jsme využili verzovací systém \texttt{Git} na privátním serveru
vedoucího člena.
\begin{verbatim}
Inkrementánlí vývojový cyklus !!!
\end{verbatim}
Konzultace probíhaly jednou týdně, obsahovaly zhodnocení aktuálních výsledků
a stanovení dalšího postupu. Nejdříve tedy každý člen pracoval sám, po několika
týdnech práce proběhly dvě schůzky na kterých jsme programovali společně.
V průběhu celého procesu členové týmu doplňovali krátké testovací kódy, kterými
bylo následně možné, pomocí skriptu napsaného v jazyce \texttt{Python}, testovat
dosavadní stabilitu celku. Při testování se nám též osvědčil program
\texttt{gprof}.

\begin{verbatim}
Něco o valgrindu?
\end{verbatim}

\newpage

\section{Implementace interpretu jazyka IFJ15} \label{implementace}
%%%%%%%%%%%%%%%%%%%%%%%%%%%%%%%%%%%%%%%%%%%%%%%%
%                   Lexikální analýza
%%%%%%%%%%%%%%%%%%%%%%%%%%%%%%%%%%%%%%%%%%%%%%%%
\subsection{Lexikální analýza} \label{lexer}

\begin{verbatim}
*** Lukáš Osadský
unget_token
o komentářích (návrat do startu)
pouívání stringu
\end{verbatim}
Lexikální analyzátor je vstupní část překladače. Je založen na deterministickém
konečném automatu, jehož hlavním úkolem je čtení zdrojového souboru a na základě
lexikálních pravidel jazyka rozdělit jednotlivé posloupnosti znaků
souboru na lexikální části  -- lexémy. Rozpoznané lexémy jsou reprezenované
strukturou token, která obsahuje informace o typu
tokenu a jeho data. Jeho vedlejší úlohou je odstraňování všech komentářů
a bílých znaků, neboť nejsou potřebné pro
následné zpracování. Princip fungování lexikálního analyzátoru reprezuntuje
příloha \ref{subsec:automat}, ve které je zobrazeno jeho schéma.
Činnost lexikálního analyzátoru je přímo řízena syntaktickým analyzátorem,
který postupně žádá o jednotlivé tokeny.

%%%%%%%%%%%%%%%%%%%%%%%%%%%%%%%%%%%%%%%%%%%%%%%%
%                   Syntaktická a sémantická analýza
%%%%%%%%%%%%%%%%%%%%%%%%%%%%%%%%%%%%%%%%%%%%%%%%
\subsection{Syntaktická a sémantická analýza} \label{parser}

\subsubsection{Zpracování jazykových konsturkcí}
Syntaktický a sémantický analyzátor, neboli \texttt{parser}, představuje ústřední
část naší implementace interpretu jazyka IFJ15. Parser se volá prakticky ihned po
spuštění programu a přejímá řízení do doby, než dojde k úplnému zpracování
zdrojového souboru.

Syntaktická analýza je implementována rekurzivním sestupem, který je řízen
pravidly naší LL-gramatiky uvedenými v příloze \ref{subsec:llgram}. Neterminální symboly představují tokeny přijaté od
lexikálního analyzátoru. Ten je volán přímo z parseru vždy, když je třeba
zpracovat další token. Se syntaktickou analýzou je současně vykonávána také
analýza sémantická. Při deklaraci nebo definici funkce -- jazyk IFJ15 podporuje v
globálním prostoru pouze funkce -- se do globální tabulky symbolů ukládá datová
struktura reprezentující danou funkci.

V případě definice funkce poté dochází ke zpracování těla dané funkce.
Přímo během rekurzivního sestupu se tak vykonávají všechny potřebné
sémantické kontroly a naplňuje se lokální tabulka symbolů. Taktéž se generují
vnitřní instrukce, které se ukládají do instrukčního seznamu příslušné funkce.
Pokud se během syntaktické analýzy narazí na výraz, je řízení programu předáno
modulu pro vyhodnocování výrazů \texttt{expr}, který pomocí precedenční analýzy
provede vyhodnocení daného výrazu a poté předá řízení zpět parseru.

Po zpracování celého zdrojového souboru se provádí závěrečné sémantické
kontroly. Kontroluje se například, zda došlo během zpracování zdrojového
souboru k definici všech deklarovaných funkcí, přesná signatura fce main.
Tímto je syntaktická a sémantická analýza ukončena a parser předá řízení interpretu.

\subsubsection{Zpracování výrazů a volání funkcí}
Zpracování výrazu je voláno v několika rozličných situacích. Existují situace,
kdy se však na místě výrazu může objevit volání funkce. Volání funkcí i zpracovávání
výrazů jsou v naší implementaci součástí jednoho modulu???.

Zpracování výrazů řízené precedenční tabulkou uvedenou v příloze
\ref{subsec:precetable} probíhá ve dvou krocích. V prvním kroku je za pomoci
zásobníkové struktury výraz převeden z infixové
na postfixovou notaci. V tomto kroku je kontrolována správná posloupnost
operátorů, operandů a závorek.

V kroku druhém je vyhodnocena postfixová notace a vygenerovány příslušné instrukce.
V této fázi běhu interpretu se kontrolují datové typy operandů a nastavují
odvozené datové typy proměnným s modifikátorem \texttt{auto}.

Při výskytu volání funkce je mimo jiné kontrolován datový typ proměnné,
kterou tato funkce nastavuje svojí návratovou hodnotou.

\subsubsection{Sémantická analýza}
Sémantická analýza probíhá pralelně se syntaktickou analýzou v rámci
rekurzivního sestupu i precedenční analýzy výrazů. Kontroluje se definice
a deklarace funkcí i deklarace proměnných.

%%%%%%%%%%%%%%%%%%%%%%%%%%%%%%%%%%%%%%%%%%%%%%%%
%                   Interpret
%%%%%%%%%%%%%%%%%%%%%%%%%%%%%%%%%%%%%%%%%%%%%%%%
\subsection{Interpret} \label{interpret}

\begin{verbatim}
*** Pavol Plaskoň
Text
\end{verbatim}
%%%%%%%%%%%%%%%%%%%%%%%%%%%%%%%%%%%%%%%%%%%%%%%%
%                   Datové struktury
%%%%%%%%%%%%%%%%%%%%%%%%%%%%%%%%%%%%%%%%%%%%%%%%
\subsection{Datové struktury}

\begin{verbatim}
Úvodní text ke sktrukturám? Vypadalo by to podle mě lépe
\end{verbatim}

\subsubsection{Zásobník}
\begin{verbatim}
*** Tvůrce??
Text
Umí přístup přes indexy, je udělanej jako pole... myslím
\end{verbatim}

\subsubsection{Řetězec}
\begin{verbatim}
*** Tvůrce??
Text
\end{verbatim}

\subsubsection{Tabulka s rozptýlenými položkami}
\begin{verbatim}
*** Pavol Plaskoň
používáme sdbm
explicitně řazené položky
\end{verbatim}
Datová struktura použitá pro tabulky symbolů. Její Výhodou je rychlost vyhledávání
položek. Základem je pole ukazatelů na
jednotlivé položky. Položky obsahují svůj klíč, data a ukazatel na další
položku, aby mohly být propojené v jednosměrně vázaný lineární
seznam -- seznam synonym. V případě ideální hashovací funkce není propojení v
seznam potřebné a čas přístupu k položkám konstantní. Nalezení takové
funkce není triviální. V případě konfliktu se čas nalezení položky
prodloužuje o dobu prohledání lineárního seznamu.


%%%%%%%%%%%%%%%%%%%%%%%%%%%%%%%%%%%%%%%%%%%%%%%%
%                   Algoritmy
%%%%%%%%%%%%%%%%%%%%%%%%%%%%%%%%%%%%%%%%%%%%%%%%
\subsection{Algoritmy}

\begin{verbatim}
Úvodní text k algoritmům?
\end{verbatim}

\subsubsection{Řadíci algoritmus -- Heap Sort}
\begin{verbatim}
*** Tvůrce??
Text
\end{verbatim}
Funkce pro seřazení prvků v poli.

\subsubsection{Vyhledávaní podřetězce -- Knuth-Morris-Pratt}
Vyhledání podřetězce v řetězci ve vestavěné funkci \texttt{find} je řešeno
algoritmem Knuth-Morris-Pratt. Základem algoritmu je vytvoření masky, tzv.
\texttt{Fail vector}. Jedná se o pole celých čísel délky hledaného textu. Ke
každému písmenu hledaného řetězce je přiřazeno číslo, které určuje index pro návrat programu
v případě neshody znaků.

\newpage
%%%%%%%%%%%%%%%%%%%%%%%%%%%%%%%%%%%%%%%%%%%%%%%%
%                   Přílohy
%%%%%%%%%%%%%%%%%%%%%%%%%%%%%%%%%%%%%%%%%%%%%%%%
\section{Přílohy} \label{prilohy}

\renewcommand\thesubsection{\thesection.\Alph{subsection}}
% AUTOMAT

% Changes after import
%	{$			{\em 
%	>			\textgreater
%	<			\textless
%	"			''
%	\			\textbackslash
%	\mbox{ }	SMAZAT
%	{}			\{\}
%	$\leq$ $\geq$
%	hex#1		hex\#1
%	hex#2		hex\#2
%	cout		
%	cin

\subsection{Diagram konečného autommlatu lexikální analýzy} \label{subsec:automat}
\begin{center}
\begin{tikzpicture}[scale=0.165]
\tikzstyle{every node}+=[inner sep=0pt]
\draw [black] (40,-55) circle (3);
\draw (40,-55) node {\em start};
\draw [black] (7,-114.5) circle (3);
\draw (7,-114.5) node {\em +*-};
\draw [black] (7,-114.5) circle (2.4);
\draw [black] (41.4,-89.7) circle (3);
\draw (41.4,-89.7) node {\em ()\{\}};
\draw [black] (41.4,-89.7) circle (2.4);
\draw [black] (7,-87.3) circle (3);
\draw (7,-87.3) node {\em =};
\draw [black] (7,-87.3) circle (2.4);
\draw [black] (77,-51.2) circle (3);
\draw (77,-51.2) node {\em /};
\draw [black] (77,-51.2) circle (2.4);
\draw [black] (94.1,-43.4) circle (3);
\draw (94.1,-43.4) node {\em lbcom};
\draw [black] (92.2,-65.3) circle (3);
\draw (92.2,-65.3) node {\em lcom};
\draw [black] (69.1,-39.7) circle (3);
\draw (69.1,-39.7) node {\em rbcom};
\draw [black] (76,-24) circle (3);
\draw (76,-24) node {\em \textless};
\draw [black] (76,-24) circle (2.4);
\draw [black] (47.8,-19.9) circle (3);
\draw (47.8,-19.9) node {\em \textgreater};
\draw [black] (47.8,-19.9) circle (2.4);
\draw [black] (77,-5.6) circle (3);
\draw (77,-5.6) node {\em cin};
\draw [black] (77,-5.6) circle (2.4);
\draw [black] (94.1,-21.3) circle (3);
\draw (94.1,-21.3) node {\em $\leq$};
\draw [black] (94.1,-21.3) circle (2.4);
\draw [black] (37.7,-9.6) circle (3);
\draw (37.7,-9.6) node {\em cout};
\draw [black] (37.7,-9.6) circle (2.4);
\draw [black] (62.2,-11.7) circle (3);
\draw (62.2,-11.7) node {\em $\geq$};
\draw [black] (62.2,-11.7) circle (2.4);
\draw [black] (6.2,-105) circle (3);
\draw (6.2,-105) node {\em eq};
\draw [black] (6.2,-105) circle (2.4);
\draw [black] (10.5,-64.6) circle (3);
\draw (10.5,-64.6) node {\em !};
\draw [black] (6.2,-54.3) circle (3);
\draw (6.2,-54.3) node {\em !=};
\draw [black] (6.2,-54.3) circle (2.4);
\draw [black] (10.5,-76.3) circle (3);
\draw (10.5,-76.3) node {\em eof};
\draw [black] (10.5,-76.3) circle (2.4);
\draw [black] (95.9,-86.5) circle (3);
\draw (95.9,-86.5) node {\em quot};
\draw [black] (78.6,-82.8) circle (3);
\draw (78.6,-82.8) node {\em str};
\draw [black] (78.6,-82.8) circle (2.4);
\draw [black] (76,-107.1) circle (3);
\draw (76,-107.1) node {\em esc};
\draw [black] (94.1,-126.4) circle (3);
\draw (94.1,-126.4) node {\em hex\#1};
\draw [black] (58.5,-82.8) circle (3);
\draw (58.5,-82.8) node {\em int};
\draw [black] (58.5,-82.8) circle (2.4);
\draw [black] (44.9,-98.5) circle (3);
\draw (44.9,-98.5) node {\em expe};
\draw [black] (69.1,-99.5) circle (3);
\draw (69.1,-99.5) node {\em int.};
\draw [black] (17,-19.9) circle (3);
\draw (17,-19.9) node {\em id};
\draw [black] (17,-19.9) circle (2.4);
\draw [black] (23.1,-7.9) circle (3);
\draw (23.1,-7.9) node {\em key};
\draw [black] (23.1,-7.9) circle (2.4);
\draw [black] (80.4,-133.6) circle (3);
\draw (80.4,-133.6) node {\em double};
\draw [black] (80.4,-133.6) circle (2.4);
\draw [black] (49,-137.1) circle (3);
\draw (49,-137.1) node {\em exp};
\draw [black] (49,-137.1) circle (2.4);
\draw [black] (29.3,-135.3) circle (3);
\draw (29.3,-135.3) node {\em exsqn};
\draw [black] (6.2,-124) circle (3);
\draw (6.2,-124) node {\em , ;};
\draw [black] (6.2,-124) circle (2.4);
\draw [black] (92.2,-109.8) circle (3);
\draw (92.2,-109.8) node {\em hex\#2};
\draw [black] (37.248,-53.835) arc (274.78927:-13.21073:2.25);
\draw (30.64,-49.4) node [above] {\em whitespace};
\fill [black] (39.25,-52.11) -- (39.68,-51.27) -- (38.69,-51.35);
\draw [black] (19.7,-55) -- (37,-55);
\fill [black] (37,-55) -- (36.2,-54.5) -- (36.2,-55.5);
\draw [black] (39.093,-57.859) arc (-19.46398:-71.76436:46.378);
\fill [black] (9.88,-86.45) -- (10.79,-86.68) -- (10.48,-85.73);
\draw (28.89,-76.03) node [below] {\em =};
\draw [black] (39.278,-57.912) arc (-14.62071:-43.4067:125.206);
\fill [black] (9.09,-112.35) -- (10,-112.11) -- (9.27,-111.42);
\draw (28.29,-88.23) node [right] {\em +*-};
\draw [black] (40.848,-57.877) arc (15.10729:-10.48651:65.261);
\fill [black] (42.01,-86.76) -- (42.65,-86.07) -- (41.67,-85.89);
\draw (43.61,-72.24) node [right] {\em ()\{\}};
\draw [black] (42.702,-53.697) arc (114.04706:77.68068:50.557);
\fill [black] (74.09,-50.47) -- (73.41,-49.81) -- (73.2,-50.79);
\draw (57.98,-49.01) node [above] {\em /};
\draw [black] (79.408,-49.413) arc (124.31794:104.72135:37.986);
\fill [black] (91.17,-44.05) -- (90.27,-43.77) -- (90.52,-44.73);
\draw (84.08,-45.72) node [above] {\em *};
\draw [black] (79.609,-52.679) arc (58.24041:36.05965:38.718);
\fill [black] (90.53,-62.81) -- (90.46,-61.87) -- (89.65,-62.46);
\draw (86.35,-56.73) node [above] {\em /};
\draw [black] (92.735,-68.24) arc (38.0546:-249.9454:2.25);
\draw (88.69,-72.4) node [below] {\em else};
\fill [black] (90.19,-67.51) -- (89.25,-67.61) -- (89.87,-68.4);
\draw [black] (95.133,-40.596) arc (187.51736:-100.48264:2.25);
\draw (99.52,-37.28) node [right] {\em else};
\fill [black] (96.95,-42.51) -- (97.81,-42.9) -- (97.68,-41.91);
\draw [black] (70.517,-37.063) arc (145.12644:18.03624:12.98);
\fill [black] (70.52,-37.06) -- (71.38,-36.69) -- (70.56,-36.12);
\draw (83.43,-31.05) node [above] {\em *};
\draw [black] (72.099,-39.642) arc (89.82687:73.3358:67.48);
\fill [black] (91.25,-42.48) -- (90.62,-41.77) -- (90.34,-42.73);
\draw (82.53,-39.68) node [above] {\em else};
\draw [black] (42.983,-55.318) arc (83.61897:74.05683:283.329);
\fill [black] (42.98,-55.32) -- (43.72,-55.9) -- (43.83,-54.91);
\draw (67.05,-58.31) node [above] {\em \textbackslash n};
\draw [black] (42.365,-53.155) arc (126.94089:108.52749:84.291);
\fill [black] (42.37,-53.15) -- (43.31,-53.07) -- (42.7,-52.27);
\draw (53.02,-45.42) node [above] {\em /};
\draw [black] (41.643,-52.49) arc (145.72712:115.7371:80.666);
\fill [black] (73.27,-25.25) -- (72.34,-25.15) -- (72.77,-26.05);
\draw (54.59,-36.3) node [above] {\em \textless};
\draw [black] (47.785,-22.9) arc (-1.39065:-23.66697:77.89);
\fill [black] (47.78,-22.9) -- (47.27,-23.69) -- (48.27,-23.71);
\draw (46.7,-38.28) node [right] {\em \textgreater};
\draw [black] (75.333,-21.077) arc (-170.18655:-196.03513:28.304);
\fill [black] (76.02,-8.43) -- (75.32,-9.06) -- (76.28,-9.34);
\draw (74.38,-14.69) node [left] {\em \textless};
\draw [black] (78.777,-22.869) arc (109.52567:87.44298:32.562);
\fill [black] (91.11,-21.03) -- (90.34,-20.49) -- (90.29,-21.49);
\draw (84.46,-20.75) node [above] {\em =};
\draw [black] (45.096,-18.61) arc (-120.5542:-150.56921:16.987);
\fill [black] (38.94,-12.33) -- (38.89,-13.27) -- (39.76,-12.78);
\draw (41.08,-17.35) node [left] {\em \textgreater};
\draw [black] (49.971,-17.831) arc (130.58707:108.73115:28.355);
\fill [black] (59.31,-12.51) -- (58.4,-12.29) -- (58.72,-13.24);
\draw (53.33,-14.22) node [above] {\em =};
\draw [black] (7.643,-90.229) arc (9.39106:-14.56682:28.726);
\fill [black] (7.1,-102.14) -- (7.79,-101.49) -- (6.82,-101.24);
\draw (8.57,-96.23) node [right] {\em =};
\draw [black] (37.384,-56.467) arc (-61.91096:-82.03681:71.936);
\fill [black] (13.48,-64.25) -- (14.34,-64.63) -- (14.2,-63.64);
\draw (26.51,-61.94) node [below] {\em !};
\draw [black] (8.605,-62.281) arc (-146.63075:-168.05055:14.588);
\fill [black] (6.52,-57.28) -- (6.19,-58.16) -- (7.17,-57.96);
\draw (6.59,-60.81) node [left] {\em =};
\draw [black] (38.338,-57.497) arc (-35.44861:-72.89027:47.931);
\fill [black] (13.39,-75.51) -- (14.3,-75.75) -- (14.01,-74.79);
\draw (29.13,-69.06) node [below] {\em eof};
\draw [black] (42.832,-55.989) arc (70.25846:50.93848:173.594);
\fill [black] (93.59,-84.59) -- (93.28,-83.7) -- (92.65,-84.47);
\draw (70.32,-67.64) node [above] {\em ''};
\draw [black] (95.246,-83.584) arc (220.37178:-67.62822:2.25);
\draw (98.93,-79.29) node [above] {\em else};
\fill [black] (97.82,-84.21) -- (98.75,-84.07) -- (98.1,-83.31);
\draw [black] (93.018,-87.318) arc (-79.48009:-124.66418:16.135);
\fill [black] (80.9,-84.73) -- (81.27,-85.59) -- (81.84,-84.77);
\draw (86.14,-87.81) node [below] {\em ''};
\draw [black] (94.409,-89.103) arc (-31.44361:-56.576:52.458);
\fill [black] (78.55,-105.52) -- (79.49,-105.5) -- (78.94,-104.66);
\draw (87.91,-99.66) node [right] {\em /};
\draw [black] (77.103,-104.311) arc (155.88922:116.09118:33.931);
\fill [black] (93.15,-87.7) -- (92.21,-87.6) -- (92.65,-88.5);
\draw (83.14,-93.13) node [left] {\em n t \ ''};
\draw [black] (91.24,-125.501) arc (-110.98777:-162.6877:24.35);
\fill [black] (91.24,-125.5) -- (90.67,-124.75) -- (90.31,-125.68);
\draw (81.67,-120.89) node [left] {\em x};
\draw [black] (42.397,-56.803) arc (51.17912:16.10584:46.023);
\fill [black] (57.76,-79.89) -- (58.02,-78.99) -- (57.06,-79.26);
\draw (52.47,-65.83) node [right] {\em isdigit};
\draw [black] (55.522,-83.049) arc (302.51909:14.51909:2.25);
\draw (51.47,-79.46) node [left] {\em isdigit};
\fill [black] (56.49,-80.59) -- (56.48,-79.64) -- (55.64,-80.18);
\draw [black] (56.54,-85.07) -- (46.86,-96.23);
\fill [black] (46.86,-96.23) -- (47.77,-95.96) -- (47.01,-95.3);
\draw (51.15,-89.2) node [left] {\em e E};
\draw [black] (60.87,-84.636) arc (48.7983:16.01079:25.048);
\fill [black] (68.45,-96.57) -- (68.71,-95.67) -- (67.75,-95.94);
\draw (66.14,-88.76) node [right] {\em .};
\draw [black] (19.806,-20.959) arc (66.76612:-0.29482:33.598);
\fill [black] (19.81,-20.96) -- (20.34,-21.73) -- (20.74,-20.82);
\draw (35.27,-32.09) node [right] {\em \_ isalpha};
\draw [black] (15.339,-22.384) arc (-6.02889:-294.02889:2.25);
\draw (10.5,-24.5) node [left] {\em \_ isalnum};
\fill [black] (14.02,-20.09) -- (13.27,-19.51) -- (13.17,-20.51);
\draw [black] (17.244,-16.916) arc (-190.7275:-223.164:14.178);
\fill [black] (20.83,-9.86) -- (19.92,-10.1) -- (20.65,-10.78);
\draw (17.84,-12.01) node [left] {\em iskeyword};
\draw [black] (78.395,-131.369) arc (-140.13542:-183.19643:41.454);
\fill [black] (78.4,-131.37) -- (78.27,-130.43) -- (77.5,-131.08);
\draw (70.1,-118.53) node [left] {\em isdigit};
\draw [black] (81.377,-136.424) arc (46.82314:-241.17686:2.25);
\draw (78.71,-141.07) node [below] {\em isdigit};
\fill [black] (78.75,-136.09) -- (77.84,-136.33) -- (78.57,-137.02);
\draw [black] (77.643,-132.417) arc (-114.55876:-154.79201:64.458);
\fill [black] (46.11,-101.24) -- (46,-102.18) -- (46.91,-101.75);
\draw (57.29,-120.1) node [below] {\em e E};
\draw [black] (48.28,-134.188) arc (-166.76078:-181.11304:131.577);
\fill [black] (48.28,-134.19) -- (48.58,-133.29) -- (47.61,-133.52);
\draw (44.87,-118.05) node [left] {\em isdigit};
\draw [black] (50.915,-134.806) arc (167.87147:-120.12853:2.25);
\draw (55.9,-133.38) node [right] {\em isdigit};
\fill [black] (51.99,-137.22) -- (52.66,-137.88) -- (52.87,-136.9);
\draw [black] (29.923,-132.366) arc (167.12944:146.92511:97.148);
\fill [black] (29.92,-132.37) -- (30.59,-131.7) -- (29.61,-131.47);
\draw (34.45,-115.15) node [left] {\em +-};
\draw [black] (32.273,-134.907) arc (95.27929:74.27941:38.224);
\fill [black] (46.15,-136.17) -- (45.51,-135.48) -- (45.24,-136.44);
\draw (39.75,-134.17) node [above] {\em isdigit};
\draw [black] (39.965,-58) arc (-1.6638:-50.53252:86.323);
\fill [black] (8.55,-122.13) -- (9.48,-122.01) -- (8.85,-121.24);
\draw (31.9,-94.56) node [right] {\em , ;};
\draw [black] (92.923,-112.711) arc (12.32751:0.73156:53.253);
\fill [black] (92.92,-112.71) -- (92.61,-113.6) -- (93.58,-113.39);
\draw (94.46,-117.91) node [right] {\em isxdigit};
\draw [black] (96.738,-89.379) arc (12.73097:-30.77723:24.511);
\fill [black] (96.74,-89.38) -- (96.43,-90.27) -- (97.4,-90.05);
\draw (97.74,-98.83) node [right] {\em isxdigit};
\end{tikzpicture}
\end{center}
\newpage


\begin{landscape}
% LL GRAMATIKA
\subsection{LL-gramatika} \label{subsec:llgram}

\subsubsection*{Část první}
\begin{verbatim}
PROG->FUNCTION_DECL PROG
PROG->eps
FUNCTION_DECL->DATA_TYPE t_identifier t_lround_bracket FUNC_DECL_PARAMS t_rround_bracket NESTED_BLOCK
DATA_TYPE->t_int
DATA_TYPE->t_double
DATA_TYPE->t_string
FUNC_DECL_PARAMS->DATA_TYPE t_identifier FUNC_DECL_PARAMS_NEXT
FUNC_DECL_PARAMS_NEXT->t_comma FUNC_DECL_PARAMS
FUNC_DECL_PARAMS->eps
FUNC_DECL_PARAMS_NEXT->eps
NESTED_BLOCK->t_lcurly_bracket NBC t_rcurly_bracket
NBC->DECL_OR_ASSIGN NBC
DECL_OR_ASSIGN->DATA_TYPE t_identifier DECL_ASSIGN t_semicolon
DECL_OR_ASSIGN->t_auto t_identifier t_assign EXPRESSION t_semicolon
DECL_ASSIGN->t_assign EXPRESSION
DECL_ASSIGN->eps
NBC->FCALL_OR_ASSIGN NBC
FCALL_OR_ASSIGN->t_identifier FOA_PART2
FOA_PART2->t_lround_bracket FUNCTION_CALL_PARAMS t_rround_bracket t_semicolon
FOA_PART2->t_assign EXPRESSION t_semicolon
HARD_VALUE->t_int_value
HARD_VALUE->t_double_value
HARD_VALUE->t_string_value
FUNCTION_CALL_PARAMS->FUNCTION_CALL_PARAM FUNCTION_CALL_PARAMS_NEXT
FUNCTION_CALL_PARAMS->eps
FUNCTION_CALL_PARAM->t_identifier
FUNCTION_CALL_PARAM->HARD_VALUE
FUNCTION_CALL_PARAMS_NEXT->t_comma FUNCTION_CALL_PARAMS
FUNCTION_CALL_PARAMS_NEXT->eps
\end{verbatim}
\newpage

\subsubsection*{Část druhá}
\begin{verbatim}
NBC->BUILTIN_CALL NBC
BUILTIN_CALL->BUILTIN_FUNC t_lround_bracket FUNCTION_CALL_PARAMS t_rround_bracket t_semicolon
BUILTIN_FUNC->token_bf_length
BUILTIN_FUNC->token_bf_substr
BUILTIN_FUNC->token_bf_concat
BUILTIN_FUNC->token_bf_find
BUILTIN_FUNC->token_bf_sort
NBC->IF_STATEMENT NBC
IF_STATEMENT->t_if t_lround_bracket EXPRESSION t_rround_bracket NESTED_BLOCK ELSE_STATEMENT
ELSE_STATEMENT->t_else NESTED_BLOCK
ELSE_STATEMENT->eps
NBC->COUT NBC
COUT->t_cout t_cout_bracket COUT_OUTPUT COUT_NEXT t_semicolon
COUT_OUTPUT->t_identifier
COUT_OUTPUT->HARD_VALUE
COUT_NEXT->t_cout_bracket COUT_OUTPUT COUT_NEXT
COUT_NEXT->eps
NBC->CIN NBC
CIN->t_cin t_cin_bracket t_identifier CIN_NEXT t_semicolon
CIN_NEXT->t_cin_bracket t_identifier CIN_NEXT
CIN_NEXT->eps
NBC->FOR_STATEMENT NBC
FOR_STATEMENT->t_for t_lround_bracket FOR_DECLARATION FOR_EXPR FOR_ASSIGN t_rround_bracket NESTED_BLOCK
FOR_DECLARATION->DATA_TYPE t_identifier DECL_ASSIGN t_semicolon
FOR_DECLARATION->t_auto t_identifier t_assign EXPRESSION t_semicolon
FOR_EXPR->EXPRESSION t_semicolon
FOR_ASSIGN->t_identifier t_assign EXPRESSION
NBC->NESTED_BLOCK NBC
NBC->RETURN
RETURN->t_return EXPRESSION t_semicolon
NBC->eps
\end{verbatim}
\end{landscape}
\newpage

% PRECEDENČNÍ TABULKA
\subsection{Precedenční tabulka} \label{subsec:precetable}

\vspace{1cm}
\begin{center}
\def\arraystretch{1.3}
\begin{tabular}{|c||c|c|c|c|c|c|c|c|c|c|c|c|c|c|}\hline
\diagbox[width=10em]{Stack}{Input}              & +  & -  & *  & /  & (  & )  & id & \textless & \textgreater & \textless= & \textgreater= & == & != & \$ \\ \hline \hline
+             & \textgreater & \textgreater & \textless & \textless & \textless & \textgreater & \textless & \textgreater        & \textgreater           & \textgreater         & \textgreater            & \textgreater & \textgreater & \\ \hline
-             & \textgreater & \textgreater & \textless & \textless & \textless & \textgreater & \textless & \textgreater        & \textgreater           & \textgreater         & \textgreater            & \textgreater & \textgreater & \\ \hline
*             & \textgreater & \textgreater & \textgreater & \textgreater & \textless & \textgreater & \textless & \textgreater        & \textgreater           & \textgreater         & \textgreater            & \textgreater & \textgreater & \\ \hline
/             & \textgreater & \textgreater & \textgreater & \textgreater & \textless & \textgreater & \textless & \textgreater        & \textgreater           & \textgreater         & \textgreater            & \textgreater & \textgreater & \\ \hline
(             & \textless & \textless & \textless & \textless & \textless & = & \textless & \textless        & \textless           & \textless         & \textless            & \textless & \textless & \\ \hline
)             & \textgreater & \textgreater & \textgreater & \textgreater & ! & \textgreater & ! & \textgreater        & \textgreater           & \textgreater         & \textgreater            & \textgreater & \textgreater & \\ \hline
id            & \textgreater & \textgreater & \textgreater & \textgreater & ! & \textgreater & ! & \textgreater        & \textgreater           & \textgreater         & \textgreater            & \textgreater & \textgreater & \\ \hline
\textless     & \textless & \textless & \textless & \textless & \textless & \textgreater & \textless & \textgreater        & \textgreater           & \textgreater         & \textgreater            & \textgreater & \textgreater & \\ \hline
\textgreater  & \textless & \textless & \textless & \textless & \textless & \textgreater & \textless & \textgreater        & \textgreater           & \textgreater         & \textgreater            & \textgreater & \textgreater & \\ \hline
\textless=    & \textless & \textless & \textless & \textless & \textless & \textgreater & \textless & \textgreater        & \textgreater           & \textgreater         & \textgreater            & \textgreater & \textgreater & \\ \hline
\textgreater= & \textless & \textless & \textless & \textless & \textless & \textgreater & \textless & \textgreater        & \textgreater           & \textgreater         & \textgreater            & \textgreater & \textgreater & \\ \hline
==            & \textless & \textless & \textless & \textless & \textless & \textgreater & \textless & \textless        & \textless           & \textless         & \textless            & \textgreater & \textgreater & \\ \hline
!=            & \textless & \textless & \textless & \textless & \textless & \textgreater & \textless & \textless        & \textless           & \textless         & \textless            & \textgreater & \textgreater & \\ \hline
\end{tabular}
\end{center}
\newpage

% INSTRUKČNÍ SADA TROJADRESNÉHO KÓDU
\subsection{Instrukční sada trojadresného kódu} \label{instrukce}


\texttt{INS\_ASSIGN} \texttt{dest}, \texttt{src1}

přiradí hodnotu proměnnej src1 do dest\\
\texttt{INS\_ADD} \texttt{dest}, \texttt{src1}, \texttt{src2}

sčítá src1 a src2, výsledek uloží do dest\\
\texttt{INS\_SUB} \texttt{dest}, \texttt{src1}, \texttt{src2}

odečítá src1 od src2, výsledek uloží do dest\\
\texttt{INS\_MUL} \texttt{dest}, \texttt{src1}, \texttt{src2}

vynásobí src1 a src2, výsledek uloží do dest\\
\texttt{INS\_DIV} \texttt{dest}, \texttt{src1}, \texttt{src2}

vydelí src2 a src1, výsledek uloží do dest\\
\texttt{INS\_EQ} \texttt{dest}, \texttt{src1}, \texttt{src2}

testuje rovnost src1 a src2, výsledek uloží do dest\\
\texttt{INS\_NEQ} \texttt{dest}, \texttt{src1}, \texttt{src2}

testuje nerovnost src1 a src2, výsledek uloží do dest\\
\texttt{INS\_GREATER} \texttt{dest}, \texttt{src1}, \texttt{src2}

testuje, je-li(hm..advanced czech?)src1 větší než src2, výsledek uloží do dest\\
\texttt{INS\_GREATEQ} \texttt{dest}, \texttt{src1}, \texttt{src2}

testuje, je-li src1 větší, nebo roven src2, výsledek uloží do dest\\
\texttt{INS\_LESSER} \texttt{dest}, \texttt{src1}, \texttt{src2}

testuje, je-li src1 menší než src2, výsledek uloží do dest\\
\texttt{INS\_LESSEQ} \texttt{dest}, \texttt{src1}, \texttt{src2}

testuje, je-li src1 menší, nebo roven src2, výsledek uloží do dest\\
\texttt{INS\_JMP} \texttt{label}

nepodmíněný skok na návěští \\
\texttt{INS\_CJMP} \texttt{cond} \texttt{label}

podmíněný skok na návěští label na základe hodnoty cond\\
\texttt{INS\_LAB} \texttt{label}

návěští pro skok\\
\texttt{INS\_PUSH\_PARAM} \texttt{src}

uloží na pomocnej zásobník parameter pro volání funkce\\
\texttt{INS\_CALL} \texttt{func}

volání funkce func\\
\texttt{INS\_RET}

ukončení provádení funkce\\
\texttt{INS\_PUSH\_TAB} \texttt{src1}

vytvoření nověho rámce pro vnořený blok src1\\
\texttt{INS\_POP\_TAB} \texttt{src1}

zrušení rámce jednoho bloku programu\\
\texttt{INS\_LENGTH} \texttt{dest}, \texttt{src1}

volání vestavěné funkce \texttt{length} s parametrem src1, výsledek uloží do dest\\
\texttt{INS\_SUBSTR} \texttt{dest}

volání vestavěné funkce \texttt{substr} s předem uloženými parametry, výsledek uloží do dest\\
\texttt{INS\_CONCAT} \texttt{dest}, \texttt{src1}, \texttt{src2}

volání vestavěné funkce \texttt{concat} s parametry src1 a src2, výsledek uloží do dest\\
\texttt{INS\_FIND} \texttt{dest}, \texttt{src1}, \texttt{src2}

volání vestavěné funkce \texttt{find} s parametry src1 a src2, výsledek uloží do dest\\
\texttt{INS\_SORT} \texttt{dest}, \texttt{src1}

volání vestavěné funcke \texttt{sort} s parametrem src1, výsledek uloží do dest\\
\texttt{INS\_CIN} \texttt{src1}

vypíše na standardní výstup src1\\
\texttt{INS\_COUT} \texttt{dest}

načítá do dest určitou? hodnotu ze standardního vstupu\\

\section{Zdroje}

\begin{verbatim}
http://madebyevan.com/fsm/
sdbm algoritmus
\end{verbatim}

\end{document}

