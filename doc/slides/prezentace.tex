%
% Course IFJ @ FIT VUT Brno, 2015
% IFJ15 Interpreter Project
%
% Authors:
% Lukas Osadsky  - xosads00
% Pavol Plaskon  - xplask00
% Pavel Pospisil - xpospi88
% Matej Postolka - xposto02
%
% Unless otherwise stated, all code is licensed under a
% Creative Commons Attribution-ShareAlike 4.0 International License
%

\documentclass[pdf]{beamer}
\usepackage[czech]{babel}
\usepackage[utf8]{inputenc}
\usepackage[IL2]{fontenc}
\usepackage{graphics}

\usepackage{xcolor,colortbl}
\usepackage{color}
\usepackage{utopia}
\usepackage{tikz}
\usetheme{Warsaw}


\setbeamertemplate{itemize item}{$\circ$}
\setbeamertemplate{itemize subitem}{$\triangleright$}

\setbeamerfont{caption}{series=\normalfont,size=\fontsize{8}{40}}
\setbeamertemplate{caption}{\raggedright\insertcaption\par}
\setbeamertemplate{navigation symbols}{}

\makeatletter
\setbeamertemplate{footline}
{
     \leavevmode%
     \hbox{%
     \begin{beamercolorbox}[wd=.5\paperwidth,ht=2.25ex,dp=1ex,center]{title in head/foot}%
            \usebeamerfont{title in head/foot}\insertshorttitle
     \end{beamercolorbox}%
     \begin{beamercolorbox}[wd=.5\paperwidth,ht=2.25ex,dp=1ex,right]{date in head/foot}%
            \usebeamerfont{date in head/foot}
			\hspace*{2em}
            \insertframenumber{} /
            \inserttotalframenumber\hspace*{2ex}
     \end{beamercolorbox}}%
     \vskip0pt%
}
\makeatother


\title{Prezentace k projektu do předmětů IFJ a IAL}
\author{Postolka Matěj, Osadský Lukáš, Plaskoň Pavol, Pospíšil Pavel}

\begin{document}

%*******************************************************************************
%				Titulni strana
%*******************************************************************************

\begin{frame}
  \titlepage
\end{frame}

%*******************************************************************************
%				Uvod
%*******************************************************************************

\begin{frame}
  \frametitle{Obsah prezentace}
  \LARGE
  \begin{itemize}
    \item Struktura programu
	\item Parser
	\item Interpretace
 \end{itemize}
\end{frame}

%*******************************************************************************
%				Struktura programu
%*******************************************************************************

\begin{frame}
  \frametitle{Struktura programu}

\begin{center}


\begin{tikzpicture}[scale=0.32, line width=0.5mm]
\tikzstyle{every node}+=[inner sep=0pt]

% ********************** main
\onslide<1-> {
	\draw [black] (15.8,-3.4) circle (3);
	\draw (15.8,-3.4) node {main};
}

% ********************** parser
\onslide<2-> {
	\draw [black] (15,-12.6) circle (3);
	\draw (15,-12.6) node {parser};
	% main to parser
	\draw [black] (13.902,-9.83) arc (-169.85004:-200.08944:7.487);
	\fill [black] (13.9,-9.83) -- (14.25,-8.95) -- (13.27,-9.13);
}

% ********************** lexer
\onslide<3-> {
	\draw [black] (5.4,-7.2) circle (3);
	\draw (5.4,-7.2) node {lexer};
	% parser to lexer
	\draw [black] (8.332,-7.805) arc (71.97254:49.31195:13.514);
	\fill [black] (8.33,-7.81) -- (8.94,-8.53) -- (9.25,-7.58);
	% lexer to parser
	\draw [black] (12.015,-12.629) arc (-99.46249:-139.25302:8.581);
	\fill [black] (12.02,-12.63) -- (11.31,-12) -- (11.14,-12.99);
}

% ********************** expr
\onslide<4-> {


\draw [black] (7,-19.3) circle (3);
\draw (7,-19.3) node {expr};

% parser to expr
\draw [black] (8.604,-16.773) arc (140.76988:119.12247:12.596);
\fill [black] (8.6,-16.77) -- (9.5,-16.47) -- (8.72,-15.84);

% expr to parser
\draw [black] (13.913,-15.378) arc (-31.75363:-68.35402:8.28);
\fill [black] (13.91,-15.38) -- (13.07,-15.8) -- (13.92,-16.32);


}

% ********************** expr to lexer
% expr to lexer + lexer to expr
\onslide<5-> {


\draw [black] (6.359,-10.041) arc (15.08386:-0.01866:24.053);
\fill [black] (6.36,-10.04) -- (6.08,-10.94) -- (7.05,-10.68);
\draw [black] (5.726,-16.589) arc (-160.3767:-184.5581:15.507);
\fill [black] (5.73,-16.59) -- (5.93,-15.67) -- (4.99,-16);


}

% ********************** instrukce
\onslide<6-> {


\draw [black, dashed] (23.4,-19.3) circle (3);
\draw (23.4,-19.3) node {instrukce};

% expr to ins
\draw [black, dashed] (10,-19.3) -- (20.4,-19.3);
\fill [black] (20.4,-19.3) -- (19.6,-18.8) -- (19.6,-19.8);

% parser to ins
\draw [black, dashed] (17.35,-14.47) -- (21.05,-17.43);
\fill [black] (21.05,-17.43) -- (20.74,-16.54) -- (20.12,-17.32);


}


% ********************** interpret
\onslide<7-> {

% parser to main
	\draw [black] (17.668,-5.697) arc (22.95489:-32.89437:5.319);
	\fill [black] (17.67,-5.7) -- (17.52,-6.63) -- (18.44,-6.24);

\draw [black] (24.3,-8.2) circle (3);
\draw (24.3,-8.2) node {interpret};

% main to interpret
\draw [black] (18.737,-2.953) arc (84.77804:36.31461:6.196);
\fill [black] (23.17,-5.45) -- (23.09,-4.51) -- (22.29,-5.11);


}

% ********************** intrukce do interpretu
\onslide<8-> {


% instrukce to interpret
\draw [black, dashed] (24.76,-11.161) arc (3.98535:-13.25627:17.708);
\fill [black] (24.76,-11.16) -- (24.32,-11.99) -- (25.31,-11.92);

}

% ********************** out
\onslide<9-> {


\draw [black, dashed] (31.3,-13.6) circle (3);
\draw (31.3,-13.6) node {výstup};

% interpret to out
\draw [black, dashed] (27.261,-8.41) arc (71.48095:33.22381:5.944);
\fill [black] (30.35,-10.79) -- (30.33,-9.85) -- (29.49,-10.39);


}

\end{tikzpicture}

\end{center}

\end{frame}

%*******************************************************************************
%				Parser
%*******************************************************************************

\begin{frame}
    \frametitle{Parser}
	\LARGE
    \begin{itemize}
       \item Rekurzivní sestup
	   \item Tabulky symbolů pro
		\begin{itemize}
			\Large
			\item funkce
			\item proměnné
			\item konstanty
		\end{itemize}
	   \item Kontroly
	   \item Zpracování výrazů
   \end{itemize}
\end{frame}

%*******************************************************************************
%				Interpretace
%*******************************************************************************

\begin{frame}
    \frametitle{Interpretace}
	\LARGE
	\begin{itemize}
		\item Zpracování instrukcí
		\item Algoritmy
		\begin{itemize}
			\item Knuth-Morris-Pratt algoritmus
			\item Heapsort
			\item Tabulka s rozptýlenými položkami
		\end{itemize}
	\end{itemize}
\end{frame}

%*******************************************************************************

\begin{frame}
    \frametitle{Interpretace}
	\framesubtitle{Zpracování instrukcí}

	\begin{centering}


	\begin{tikzpicture}[scale=0.65, line width=0.3mm]
	\scriptsize

	\tikzstyle{every node}+=[inner sep=0pt]

	% Aktivni ramec
	\draw (1,0) rectangle (4,3);
	\draw (1,3) rectangle (4,4);
	\draw (1,4) rectangle (4,5);

	\draw (0, 4.5) node {\begin{tabular}{c} Aktivní\\rámec \end{tabular}};
	\draw (2.5, 4.5) node {tělo funkce \texttt{f1}};
	\draw (2.5, 3.5) node {složený příkaz};

	\draw (4,4.5) -- (4.9,4.5);
	\fill (4.6,4.7) -- (4.6,4.3) -- (5,4.5);

	\draw (4,3.5) -- (4.9,3.5);
	\fill (4.6,3.7) -- (4.6,3.3) -- (5,3.5);


	\draw (5,3) rectangle (8,4);
	\draw (5,4) rectangle (8,5);
	\draw (6.5, 4.5) node {hash tabulka};
	\draw (6.5, 3.5) node {hash tabulka};

	% Globalni ramec
	\draw (1,6) rectangle (4,9);
	\draw (1,9) rectangle (4,10);
	\draw (1,10) rectangle (4,11);

	\draw (0, 10.5) node {\begin{tabular}{c} Globální\\rámec \end{tabular}};
	\draw (2.5, 10.5) node {\texttt{main}};
	\draw (2.5, 9.5) node {\texttt{ins\_main}};

	\draw (4,10.5) -- (8.9,10.5);
	\fill (8.6,10.7) -- (8.6,10.3) -- (9,10.5);

	\draw (4,9.5) -- (4.9,9.5);
	\fill (4.6,9.7) -- (4.6,9.3) -- (5,9.5);

	\draw (5,9) rectangle (8,10);
	\draw (6.5, 9.5) node {\texttt{INS\_CALL f1}};

	\draw (9,6) rectangle (12,10);
	\draw (9,10) rectangle (12,11);

	\draw (10.5, 10.5) node {tělo fce \texttt{main}};

	\draw (12,10.5) -- (12.9,10.5);
	\fill (12.6,10.7) -- (12.6,10.3) -- (13,10.5);

	\draw (13,10) rectangle (16,11);
	\draw (14.5,10.5) node {hash tabulka};

\end{tikzpicture}

\end{centering}

\end{frame}

%*******************************************************************************

\begin{frame}[fragile]
    \frametitle{Interpretace}
	\framesubtitle{Knuth-Morris-Pratt algoritmus}


%***************************** fail vector

\begin{table}[]

\caption{Fail vector}
\begin{tabular}{r|l|l|l|l|l|l|l|l}
index pole & 1 & 2 & 3 & 4 & 5 & 6 & 7 & 8 \\ \hline
podřetězec & A & B & A & A & B & D & C & A \\ \hline
\only<1>{index skoku & 0 & 1 & 1 & 2 & 2 & 3 & 1 & 1}\only<2>{index skoku &\color{red} 0 & 1 & 1 & 2 & 2 & 3 & 1 & 1}\only<3>{index skoku & 0 & 1 & 1 &\color{red} 2 & 2 & 3 & 1 & 1}\only<4>{index skoku & 0 &\color{red} 1 & 1 & 2 & 2 & 3 & 1 & 1}\only<5>{index skoku &\color{red} 0 & 1 & 1 & 2 & 2 & 3 & 1 & 1}\only<6>{index skoku & 0 & 1 & 1 & 2 & 2 & 3 & 1 & 1}
\\
\end{tabular}

\end{table}


%****************************** ukazka
\onslide<2-> {
\begin{table}[]
\centering
\caption{Ukázka}
\begin{tabular}{l}
\tt
\Large
BABAADABAABDCABCDABA\\
\tt\Large
\only<2>{ABAABDCA}
\only<3>{ABAABDCA}
\only<4>{~~ABAABDCA}
\only<5>{~~ABAABDCA}
\only<6>{~~ABAABDCA}
\end{tabular}
\end{table}

}


\end{frame}

%*******************************************************************************

\begin{frame}



    \frametitle{Interpretace}
	\framesubtitle{Heapsort}

% ***************************** strom
	\begin{center}

\begin{tikzpicture}[scale=0.15 , line width=0.3mm]
\tikzstyle{every node}+=[inner sep=0pt]
\draw [black] (36.1,-5.2) circle (3);
\draw (36.1,-5.2) node {$1$};
\draw [black] (23.9,-13.5) circle (3);
\draw (23.9,-13.5) node {$2$};
\draw [black] (48.7,-13.5) circle (3);
\draw (48.7,-13.5) node {$3$};
\draw [black] (17.5,-21.9) circle (3);
\draw (17.5,-21.9) node {$4$};
\draw [black] (29.9,-21.9) circle (3);
\draw (29.9,-21.9) node {$5$};
\draw [black] (42.3,-21.9) circle (3);
\draw (42.3,-21.9) node {$6$};
\draw [black] (54.8,-21.9) circle (3);
\draw (54.8,-21.9) node {$7$};
\draw [black] (33.62,-6.89) -- (26.38,-11.81);
\fill [black] (26.38,-11.81) -- (27.32,-11.78) -- (26.76,-10.95);
\draw [black] (22.08,-15.89) -- (19.32,-19.51);
\fill [black] (19.32,-19.51) -- (20.2,-19.18) -- (19.41,-18.57);
\draw [black] (25.64,-15.94) -- (28.16,-19.46);
\fill [black] (28.16,-19.46) -- (28.1,-18.52) -- (27.28,-19.1);
\draw [black] (38.61,-6.85) -- (46.19,-11.85);
\fill [black] (46.19,-11.85) -- (45.8,-10.99) -- (45.25,-11.83);
\draw [black] (46.88,-15.89) -- (44.12,-19.51);
\fill [black] (44.12,-19.51) -- (45,-19.18) -- (44.21,-18.57);
\draw [black] (50.46,-15.93) -- (53.04,-19.47);
\fill [black] (53.04,-19.47) -- (52.97,-18.53) -- (52.16,-19.12);
\end{tikzpicture}
\end{center}
%********************************* pole

\begin{table}[]
\centering
\caption{Umístění uzlů v poli}
\label{my-label}
\begin{tabular}{|l|l|l|l|l|l|l|}
\hline
1 & 2 & 3 & 4 & 5 & 6 & 7 \\ \hline
\end{tabular}
\end{table}





\end{frame}

%*******************************************************************************

\begin{frame}
    \frametitle{Interpretace}
	\framesubtitle{Tabulka s rozptýlenými položkami}
	\LARGE

%*************************** hash tabulka
\begin{centering}

	\begin{tikzpicture}[scale=0.9, line width=0.3mm]

	\tikzstyle{every node}+=[inner sep=0pt]

	\onslide<1-> {
	\draw (1,1) rectangle (2,2);
	\draw (1,2) rectangle (2,3);
	\draw (1,3) rectangle (2,4);
	\draw (1,4) rectangle (2,5);
	\draw (1,5) rectangle (2,6);
	\draw (1,6) rectangle (2,7);

	\draw (1.5, 0.5) node {\vdots};
	\draw (1.5, 1.5) node {6};
	\draw (1.5, 2.5) node {5};
	\draw (1.5, 3.5) node {4};
	\draw (1.5, 4.5) node {3};
	\draw (1.5, 5.5) node {2};
	\draw (1.5, 6.5) node {1};


	%1 a
	\draw (2,6.5) -- (2.8,6.5);
	\fill (2.7,6.7) -- (2.7,6.3) -- (3,6.5);
	% 3 b
	\draw (2,4.5) -- (2.8,4.5);
	\fill (2.7,4.7) -- (2.7,4.3) -- (3,4.5);

	% 6 d
	\draw (2,1.5) -- (2.8,1.5);
	\fill (2.7,1.7) -- (2.7,1.3) -- (3,1.5);

	\draw [black] (3.5,6.5) circle (0.5);
	\draw (3.5,6.5) node {a};

	\draw [black] (3.5,4.5) circle (0.5);
	\draw (3.5,4.5) node {b};



	\draw [black] (3.5,1.5) circle (0.5);
	\draw (3.5,1.5) node {d};
	}

	\onslide<2-> {
		%3 c
	\draw (4,4.5) -- (4.8,4.5);
	\fill (4.7,4.7) -- (4.7,4.3) -- (5,4.5);

	\draw [black] (5.5,4.5) circle (0.5);
	\draw (5.5,4.5) node {c};

	}

\end{tikzpicture}

\end{centering}

%**************************************


\end{frame}

%*******************************************************************************

\end{document}
