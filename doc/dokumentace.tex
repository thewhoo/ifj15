%
% Course IFJ @ FIT VUT Brno, 2015
% IFJ15 Interpreter Project
%
% Authors:
% Lukas Osadsky  - xosads00
% Pavol Plaskon  - xplask00
% Pavel Pospisil - xpospi88
% Matej Postolka - xposto02

\documentclass[a4paper, 12pt]{article}

\usepackage[left=2cm,text={17cm, 24cm},top=3cm]{geometry}
\usepackage[czech]{babel}
\usepackage[utf8]{inputenc}
\usepackage[IL2]{fontenc}
\usepackage{times}
\usepackage{graphics}
\usepackage{url}
\usepackage{tikz}
\usetikzlibrary{automata,positioning}
\pagenumbering{arabic}

\providecommand{\uv}[1]{\quotedblbase #1\textquotedblleft}

\begin{document}
%%%%%%%%%%%%%%%%%%%%%%%%%%%%%%%%%%%%%%%%%%%%%%%%
%               Title page
%%%%%%%%%%%%%%%%%%%%%%%%%%%%%%%%%%%%%%%%%%%%%%%%
\begin{titlepage}

\begin{center}
\fontsize{25}{20}\selectfont{\textsc{Vysoké učení technické v~Brně}}\\
\vspace{\stretch{0.0075}}
\fontsize{21}{0}\textsc{\selectfont{Fakulta informačních technologií}}\\
\vspace{\stretch{0.15}}
%Logo
\begin{figure}[ht]
    \begin{center}
        \scalebox{1}{\includegraphics{logo_bw.eps}}
    \end{center}
\end{figure}

\vspace{\stretch{0.2}}
\LARGE{Dokumentace IFJ15}\\
\Large{Tým \textit{052}, varianta \textit{a/2/II}}
\vspace{\stretch{0.618}}
\end{center}

\begin{large}
\begin{tabbing}
    Další členové: \quad \= Postolka Matej \quad \= xposto02 \quad \= 25 \%\kill
    Vedoucí týmu:  \> Postolka Matej \> xposto02 \> 25 \% \\
    Další členové: \> Osadský Lukáš  \> xosads00 \> 25 \% \\
             \> Plaskoň Pavol  \> xplask00 \> 25 \% \\
             \> Pospíšil Pavel \> xpospi88 \> 25 \% \\
\end{tabbing}
\end{large}

\end{titlepage}

%%%%%%%%%%%%%%%%%%%%%%%%%%%%%%%%%%%%%%%%%%%%%%%%
%                   Obsah
%%%%%%%%%%%%%%%%%%%%%%%%%%%%%%%%%%%%%%%%%%%%%%%%
\tableofcontents
\newpage
%%%%%%%%%%%%%%%%%%%%%%%%%%%%%%%%%%%%%%%%%%%%%%%%
%                   uvod
%%%%%%%%%%%%%%%%%%%%%%%%%%%%%%%%%%%%%%%%%%%%%%%%
\section{Práce v týmu}
+ vývojový cyklus? speciální techniky
\section{Lexikální analyzátor} \label{lexer}
[Diagram konečného automatu lexikálního analyzátoru]
Challenge? (přepis automatu pomocí latex balíčků)
zdroj: http://tex.stackexchange.com/questions/20784/which-package-can-be-used-to-draw-automata
\begin{tikzpicture}[shorten >=1pt,node distance=2cm,on grid,auto] 
   \node[state,initial] (q_0)   {$q_0$}; 
   \node[state] (q_1) [above right=of q_0] {$q_1$}; 
   \node[state] (q_2) [below right=of q_0] {$q_2$}; 
   \node[state,accepting](q_3) [below right=of q_1] {$q_3$};
    \path[->] 
    (q_0) edge  node {0} (q_1)
          edge  node [swap] {1} (q_2)
    (q_1) edge  node  {1} (q_3)
          edge [loop above] node {0} ()
    (q_2) edge  node [swap] {0} (q_3) 
          edge [loop below] node {1} ();
\end{tikzpicture}

\section{Syntaktický analyzátor} \label{parser}
[LL gramatika]
Super práce to opisovat :D
\begin{verbatim}
IFJ15 LL GRAMMAR RULES
<PROG>=<FUNCTION_DECL><PROG>
?<PROG>=\varepsilon (může být prázdný soubor syntakticky správně?)
<FUNCTION_DECL>=<DATA_TYPE>t_identifier(<FUNC_DECL_PARAMS>)<NESTED_BLOCK>
<DATA_TYPE>t_int
<DATA_TYPE>t_double
<DATA_TYPE>t_string
<DATA_TYPE>t_auto
<FUNC_DECL_PARAMS>
\end{verbatim}
\section{Interpret} \label{interpret}

\subsection{Řadíci algoritmus -- Heap Sort}
Funkce pro seřazeni prvků v poli.

\subsection{Vyhledávaní podřeťezce -- Knuth-Morris-Pratt}
Vyhledáni podreťezce v řeťezci ve vestavěné funkci \texttt{find} je řešeno
algoritmom Knuth-Morris-Pratt. Základem algoritmu je vytvorení masky, tzv.
\texttt{Fail vector}. Je to pole celých čísel o délke hledaného textu. Ke
každému písmenu hledaného řeťezce je přirazeno číslo, které určuje index, kam
se má program vrátit v prípade nezhody znaků.

\subsection{Tabulka s rozptylenými položkami}
Datová struktúra použitá pro tabulky symbolů. Výhodou je rychlost vzhledávaní
položek. Základem je pole ukazatelů na
jednotlivé položky. Položky obsahují svůj klíč, data a ukazatel na dalši
položku, aby mohli být propojené v jednosmerne vázaný lineárni
seznam (seznam synonym). V prípade ideálni hashovací funkce není propojení v
seznam potřebné, čas prístupu k položkám konstantní. Nalezení takové
funkce je ale problematické. V prípade konfliktů se čas nalezení položky
prodloužuje o prohledání lineárního seznamu.




\section{Trololo} \label{trololo}
+kontrola čitelnosti a komentovanosti kódů, zdroje! (např. odkud prvočísla do hash tabulky)
\end{document}
